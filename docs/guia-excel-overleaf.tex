\documentclass[11pt,a4paper]{article}
\usepackage[spanish]{babel}
\usepackage[utf8]{inputenc}
\usepackage[T1]{fontenc}
\usepackage{geometry}
\usepackage{longtable}
\usepackage{array}
\usepackage{hyperref}
\geometry{margin=2.5cm}

\title{Guía de Llenado de Excel para Importación de Datos}
\author{J\&D Consultores de Negocios}
\date{\today}

\begin{document}
\maketitle

\section*{Objetivo}
Este documento explica exactamente qué columnas debe completar en cada hoja del Excel, con el tipo de dato, si es obligatorio y \textbf{por qué se solicita}. Siga el \textbf{orden de llenado recomendado} para evitar errores.

\subsection*{Recomendaciones generales}
\begin{itemize}
  \item Formato de fecha: \texttt{YYYY-MM-DD}. Para meses de servicio, use el día 01 (p.ej., \texttt{2024-06-01}).
  \item N\'umeros: use punto decimal (p.ej., \texttt{500.00}).
  \item Booleanos: \texttt{TRUE}/\texttt{FALSE}.
  \item IDs: cuando se pida un \texttt{IdX}, debe existir previamente o cargarse en otra hoja.
\end{itemize}

\section{Orden recomendado de llenado}
\begin{enumerate}
  \item Catálogos (opcional): Bancos, Servicios, Clasificaciones, Categorías.
  \item Carteras.
  \item Clientes.
  \item Pagos (y \emph{DetallePagoServicio} si un pago cubre varios meses/servicios).
  \item Compromisos (opcional).
  \item Notificaciones (opcional).
  \item Servicios Adicionales (opcional).
\end{enumerate}

\section{Hojas del Excel}

\subsection{Carteras}
\begin{longtable}{|p{3.2cm}|p{2.2cm}|p{2.2cm}|p{4.8cm}|}
\hline
\textbf{Campo} & \textbf{Tipo} & \textbf{Obligatorio} & \textbf{Por qué se solicita} \\
\hline
Nombre & Texto & Sí & Identifica la cartera/comunidad de clientes. Permite segmentar gestión. \\
IdEncargado & Número & No & Vincula un responsable (usuario) a la cartera. \\
Estado & Texto & No & Estado operativo (p.ej., ACTIVA/INACTIVA). \\
\hline
\end{longtable}

\subsection{Clientes}
\begin{longtable}{|p{3.2cm}|p{2.2cm}|p{2.2cm}|p{6.0cm}|}
\hline
\textbf{Campo} & \textbf{Tipo} & \textbf{Obligatorio} & \textbf{Por qué se solicita} \\
\hline
RazonSocial & Texto & Sí & Identificación principal del cliente. \\
NombreContacto & Texto & No & Contacto para comunicaciones. \\
RucDni & Texto & Sí & Identificador único. Permite validar y calcular último dígito de RUC. \\
IdClasificacion & Número & No & Segmentación (A/B/C/...). Si omite, se usa la inicial. \\
IdCartera & Número & No & Relación con cartera para gestión. \\
IdEncargado & Número & No & Asigna un responsable del cliente. \\
IdServicio & Número & No & Servicio principal contratado. \\
MontoFijoMensual & Número & No & Base de facturación/seguimiento mensual. \\
AplicaMontoFijo & Boolean & No & Indica si se usa el monto fijo para meses. \\
IdCategoriaEmpresa & Número & No & Segmenta por tamaño/sector. \\
Email & Texto & No & Envío de notificaciones/email. \\
Telefono & Texto & No & Envío de WhatsApp/SMS. \\
FechaRegistro & Fecha & No & Trazabilidad de altas. \\
FechaVencimiento & Fecha & No & Gestión de renovación/fin de servicio. \\
\hline
\end{longtable}

\subsection{Pagos}
\begin{longtable}{|p{3.2cm}|p{2.2cm}|p{2.2cm}|p{6.0cm}|}
\hline
\textbf{Campo} & \textbf{Tipo} & \textbf{Obligatorio} & \textbf{Por qué se solicita} \\
\hline
IdCliente & Número & Sí & Determina a quién corresponde el pago. \\
Fecha & Fecha & No & Fecha de operación; por defecto, ahora. \\
IdBanco & Número & No & Para reportes y conciliación bancaria. \\
Monto & Número & Sí & Importe del pago. \\
Concepto & Texto & Sí & Contexto del pago (p.ej. mes cubierto). \\
MedioPago & Texto & Recomendado & Estadística/conciliación por canal (TRANSFERENCIA, EFECTIVO, etc.). \\
UrlComprobante & Texto & No & Evidencia del pago para auditoría. \\
MesServicio & Fecha & Recomendado & Si cubre un mes fijo, indique el mes (\texttt{YYYY-MM-01}). \\
Estado & Texto & No & Flujo del pago (PENDIENTE/CONFIRMADO/CANCELADO). \\
Observaciones & Texto & No & Nro operación u otros detalles. \\
\hline
\end{longtable}

\subsection{DetallePagoServicio (opcional)}
\begin{longtable}{|p{3.2cm}|p{2.2cm}|p{2.2cm}|p{6.0cm}|}
\hline
\textbf{Campo} & \textbf{Tipo} & \textbf{Obligatorio} & \textbf{Por qué se solicita} \\
\hline
IdPago & Número & Sí & Identifica el pago al que pertenece el detalle. \\
MesServicio & Fecha & Sí & Mes cubierto (\texttt{YYYY-MM-01}). \\
Servicio & Texto & Recomendado & Servicio cubierto por ese pago. \\
Monto & Número & Sí & Porción del pago asignada al mes/servicio. \\
IdServicioAdicional & Número & No & Si el detalle paga un servicio adicional. \\
TipoServicio & Texto & No & FIJO/ADICIONAL para análisis. \\
PeriodoServicio & Fecha & No & Equivalente a MesServicio en algunos flujos. \\
\hline
\end{longtable}

\subsection{Compromisos (opcional)}
\begin{longtable}{|p{3.2cm}|p{2.2cm}|p{2.2cm}|p{6.0cm}|}
\hline
\textbf{Campo} & \textbf{Tipo} & \textbf{Obligatorio} & \textbf{Por qué se solicita} \\
\hline
IdCliente & Número & Sí & Cliente que asume el compromiso. \\
FechaCompromiso & Fecha & Sí & Control de vencimientos del compromiso. \\
MontoCompromiso & Número & Sí & Importe comprometido. \\
IdResponsable & Número & No & Seguimiento por usuario del equipo. \\
Observaciones & Texto & No & Condiciones, acuerdos o notas. \\
Estado & Texto & No & PENDIENTE/CONFIRMADO/CANCELADO. \\
\hline
\end{longtable}

\subsection{Notificaciones (opcional)}
\begin{longtable}{|p{3.2cm}|p{2.2cm}|p{2.2cm}|p{6.0cm}|}
\hline
\textbf{Campo} & \textbf{Tipo} & \textbf{Obligatorio} & \textbf{Por qué se solicita} \\
\hline
IdCliente & Número & Sí & Destinatario de la notificación. \\
IdTipoNotificacion & Número & Sí & Canal (WhatsApp/Email/SMS) definido en catálogos. \\
Contenido & Texto & Sí & Mensaje enviado al cliente. \\
IdResponsable & Número & No & Usuario que ejecuta el envío. \\
Estado & Texto & No & ENVIADO/PENDIENTE/ERROR (histórico/operativo). \\
FechaEnvio & Fecha & No & Para trazabilidad histórica. \\
\hline
\end{longtable}

\subsection{Servicios Adicionales (opcional)}
\begin{longtable}{|p{3.2cm}|p{2.2cm}|p{2.2cm}|p{6.0cm}|}
\hline
\textbf{Campo} & \textbf{Tipo} & \textbf{Obligatorio} & \textbf{Por qué se solicita} \\
\hline
IdCliente & Número & Sí & Cliente que recibió el servicio adicional. \\
NombreServicio & Texto & Sí & Identifica el servicio adicional. \\
Descripcion & Texto & No & Detalle del servicio prestado. \\
Monto & Número & Sí & Importe del servicio. \\
Fecha & Fecha & Sí & Fecha del servicio. \\
Estado & Texto & No & PENDIENTE/FACTURADO/PAGADO/CANCELADO. \\
IdResponsable & Número & No & Usuario que registra/gestiona el servicio. \\
\hline
\end{longtable}

\section*{Plantillas CSV}
Se incluyen plantillas en la carpeta \texttt{docs/plantillas/}: \texttt{carteras.csv}, \texttt{clientes.csv}, \texttt{pagos.csv}, \texttt{detalle\_pago\_servicio.csv}, \texttt{compromisos.csv}, \texttt{notificaciones.csv}, \texttt{servicios\_adicionales.csv}.

\section*{Cómo usar este documento en Overleaf}
\begin{enumerate}
  \item Cree un proyecto en Overleaf y cargue este archivo (\texttt{guia-excel-overleaf.tex}).
  \item Compile (XeLaTeX o pdfLaTeX). Descargue el PDF para entregarlo a su equipo.
  \item Entregue también las plantillas CSV para que sean completadas.
\end{enumerate}

\end{document}


